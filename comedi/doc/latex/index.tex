index: \begin{DoxyItemize}
\item \hyperlink{index_sectionCommandLine}{command line options} \item \hyperlink{index_sectionParameterFile}{CDL parameter file} \item \hyperlink{index_sectionDAQlmlDocumentation}{documentation outline} \item \hyperlink{index_sectionDoxygenSyntax}{make documentation using Doxygen syntax}\end{DoxyItemize}
\hypertarget{index_sectionCommandLine}{}\section{command line options}\label{index_sectionCommandLine}

\begin{DoxyVerbInclude}
\end{DoxyVerbInclude}
\hypertarget{index_sectionParameterFile}{}\section{CDL parameter file}\label{index_sectionParameterFile}
Use NetCDF library tool {\ttfamily ncgen} to compile {\ttfamily parameters.cdl} text file to {\ttfamily parameters.nc} binary file (e.g. {\ttfamily ncgen} {\ttfamily parameters.cdl} {\ttfamily -\/o} {\ttfamily parameters.nc}). 
\begin{DoxyVerbInclude}
netcdf HPIVlml.parameters {//written in NetCDF CDL language
dimensions:
//  dim_string=64;
  dim_plane_number=unlimited;

//variable declaration and attributes
variables:
//acquisition
  int acquisition;
    acquisition:channels= 0, 1;
    acquisition:channel_name= "pressure", "hot_wire";

//reconstructed plane parameters
  float z_init;
  float z_step;
  int   plane_number;
  //or several user defined positions
  float z_position(dim_plane_number);
//convolution reconstruction parameters
  float laser_wave_length;
  float z_reference;
  float pixel_size;

//data value
data:
//acquisition
  acquisition=1;

//reconstructed plane parameters
  z_init=0.1;
  z_step=0.2;
  plane_number=4;
  //or
  z_position=0.1, 0.2, 0.25, 0.3, 0.4;
//convolution reconstruction parameters
  laser_wave_length=532e-9;
  z_reference=1.1;
  pixel_size=6.7e-6;
}

\end{DoxyVerbInclude}
\hypertarget{index_sectionDAQlmlDocumentation}{}\section{documentation outline}\label{index_sectionDAQlmlDocumentation}
This is the reference documentation of \href{http://www.meol.cnrs.fr/}{\tt HPIVlml}, from the \href{http://www.univ-lille1.fr/lml/}{\tt LML}.\par
\par
 DAQlml is a data acquisition software. The program begins in the main function in the \href{main1_8cpp.html}{\tt main.cpp} source file.\par
\par
 This documentation has been automatically generated from the HPIVlml sources, using the tool \href{http://www.doxygen.org}{\tt doxygen}. It should be readed as HTML, LaTex and man page.\par
 It contains both \begin{DoxyItemize}
\item a detailed description of all classes and functions \item a user guide (cf. related pages)\end{DoxyItemize}
that as been documented in the sources.

\begin{DoxyParagraph}{Additional needed libraries:}

\end{DoxyParagraph}
\begin{DoxyItemize}
\item \href{http://cimg.sourceforge.net}{\tt the CImg Library} using \href{http://www.imagemagick.org/}{\tt ImageMagick} for a few image format \item \href{http://www.fftw.org/}{\tt FFTw} (Fastest Fourier Transform in the West) though CImg library \item \href{http://www.unidata.ucar.edu/software/netcdf/}{\tt NetCDF} (network Common Data Form)\end{DoxyItemize}
\begin{DoxyParagraph}{Optional libraries:}

\end{DoxyParagraph}
\begin{DoxyItemize}
\item added to CImg raw, \href{http://www.rd-vision.com/}{\tt Hiris}, \href{http://www.pco.de/}{\tt PCO} and \href{http://www.lavision.de}{\tt LaVision} images support \item \href{http://www.libpng.org/}{\tt libPNG} (Portable Network Graphics) using \href{http://www.zlib.net/}{\tt zLib} (non destructive compression) \item \href{http://www.libtiff.org/}{\tt libTIFF} (Tag Image File Format) \begin{Desc}
\item[\hyperlink{todo__todo000004}{Todo}]add MPI and ParaView \end{Desc}
\item parallelism \href{http://www.open-mpi.org/}{\tt MPI} (Message Passing Interface) \item data format \href{http://www.paraview.org/}{\tt ParaView} \begin{Desc}
\item[\hyperlink{todo__todo000005}{Todo}]may look at \end{Desc}
\item vector graphics \href{http://libboard.sourceforge.net/}{\tt Board} (A vector graphics C++ library: Postscript, SVG and Fig files) \item parallelism \href{http://www.gnu.org/software/pth/}{\tt pThread} (POSIX thread) \item \href{http://www.boost.org/}{\tt Boost.org} (aims to C++ Standard Library) \href{http://www.boost.org/libs/libraries.htm}{\tt libs} \item \href{http://www.oonumerics.org/blitz/}{\tt Blitz++} (C++ class library for scientific computing) \item \href{http://math.nist.gov/tnt/}{\tt TNT} (Template Numerical Toolkit = PACK of \href{http://math.nist.gov/lapack++/}{\tt Lapack++}, \href{http://math.nist.gov/sparselib++/}{\tt Sparselib++}, \href{http://math.nist.gov/iml++/}{\tt IML++}, and \href{http://math.nist.gov/mv++/}{\tt MV++}) , ...\end{DoxyItemize}
\hypertarget{index_sectionDoxygenSyntax}{}\section{make documentation using Doxygen syntax}\label{index_sectionDoxygenSyntax}
Each function in the source code should be commented using {\bfseries doxygen} {\bfseries syntax} in the same file. The documentation need to be written before the function. The basic syntax is presented in this part. 
\begin{DoxyVerbInclude}
\end{DoxyVerbInclude}


Two kind of comments are needed for {\bfseries declaration} and {\bfseries explanation} {\bfseries parts} of the function: Standart documentation should the following ({\bfseries sample} of code documentation): 
\begin{DoxyVerbInclude}
\end{DoxyVerbInclude}


In both declaration and explanation part, {\bfseries writting} and {\bfseries highlithing} syntax can be the following:\par
\par


\begin{DoxyItemize}
\item {\ttfamily $\backslash$n} a new line \item {\ttfamily $\backslash$li} a list (dot list)\end{DoxyItemize}
\begin{DoxyItemize}
\item {\ttfamily $\backslash$b} bold style \item {\ttfamily $\backslash$c} code style \item {\ttfamily $\backslash$e} enhanced style (italic)\end{DoxyItemize}
For making {\bfseries shortcut} please use:\par
 \begin{DoxyItemize}
\item {\ttfamily $\backslash$see} to make a shortcut to a related function or variable \item {\ttfamily $\backslash$link} to make a shortcut to a file or a function \begin{DoxyNote}{Note}
this keyword needs to be closed using {\ttfamily $\backslash$end$\ast$} 
\end{DoxyNote}
While coding or debugging, please use intensively: \item {\ttfamily $\backslash$todo} to add a thing to do in the list of \href{todo.html}{\tt ToDo} for the whole program \item {\ttfamily $\backslash$bug} to add an {\itshape a\/} {\itshape priori\/} or known bug in the list of \href{bug.html}{\tt Bug} for the whole program\end{DoxyItemize}
In explanation part, {\bfseries paragraph} style can be the following:\par
 \begin{DoxyItemize}
\item {\ttfamily $\backslash$code} for an example of the function use \item {\ttfamily $\backslash$note} to add a few notes \item {\ttfamily $\backslash$attention} for SOMETHING NOT FULLY DEFINED YET \item {\ttfamily $\backslash$warning} to give a few warning on the function \begin{DoxyNote}{Note}
these keywords need to be closed using {\ttfamily $\backslash$end$\ast$} 
\end{DoxyNote}

\begin{DoxyVerbInclude}
\end{DoxyVerbInclude}
\end{DoxyItemize}
Many other keywords are defined, so please read the documentation of \href{http://www.doxygen.org/commands.html}{\tt doxygen}. 